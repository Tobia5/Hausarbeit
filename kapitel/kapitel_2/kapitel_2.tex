\newpage
\section{Anforderungsermittlung} \label{infos}
Die \ac{SPA} soll im täglichen Betrieb über verschiedene Endgeräte eine Übersicht über aktuelle Aufgaben geben.

\subsection{Navigation}
Der Nutzer hat 4 verschiedene Navigationsbereiche.
Diese Trennung erlaubt laut dem Gesetzes der gemeinsamen Region, eine klare Trennung zwischen den unterschiedlichen Navigationsmenüs.
Die Headernavigation, enthält auf der linken Seite die Steuerung für die Seitennavigation und das Logo.
Auf der rechten Seite befinden sich Elemente zur allgemeinen Steuerung des Programmes.
Die Seitennavigation an sich stellt die Hauptnavigation der Webseite dar.
Neben dieser Art der Navigation gibt es noch eine Subnavigation mit Steuerelementen für die aktuelle Seite der \ac{SPA} und eine Footernavigation wo sich das Impressum, die Uhrzeit und der aktuelle Versionsstand der Webseite befindet.

\subsubsection{Headernavigation}

Wie bereits beschrieben dient die Headernavigation zu der allgemeinen Steuerung der \ac{SPA}.
Auf der linken Seite befindet sich ein Icon, das die Seitennavigation schließen und öffnen kann.
Nebenan ist das Icon der Webseite, welches den Nutzer auf sein Dashboard, und damit seine Hauptseite zu bringen.




Sie ist responsiv und wird ab dem BReakpoint kleiner Medium in die Seitennavigation, durch eine Pipe getrennt, unter die Hauptnavigation integriert.
Bei der Layoutgröße Medium werden nur die Icons der Buttons angezeigt, um die Headernavigation nicht zu überladen.
Hierbei werden Tooltips für die Icons mit dem Namen

